\documentclass[
  lang=cn,           % 目前只支持中文
  %degree=doctor,   % 博士论文
  %degree=master,   % 硕士论文(学硕不能使用techmaster参数)
  %techmaster,      % 专业学位硕士论文(必须与degree=master参数共同作用)
  degree=bachelor,  % 本科生论文(设计)
  type=paper,        % 支持paper(论文)或design(设计)(该参数仅对本科生作用)
  %type=design,        % 支持paper(论文)或design(设计)(该参数仅对本科生作用)
  % openany,oneside  % 单面打印
  openright,blankleft,twoside % 双面打印
]{nwafuthesis}

% 载入需要的宏包
% 请按自己的论文排版需求,加载需要的宏包

\usepackage{subfig}
\usepackage{rotating}
\usepackage[usenames,dvipsnames]{xcolor}
\usepackage{tikz}
\usepackage{pgfplots}
\pgfplotsset{compat=1.16}
\pgfplotsset{
  table/search path={./figs/},
}
\usepackage{ifthen}
\usepackage{longtable}
\usepackage{siunitx}
\usepackage{listings}
\usepackage{multirow}
\usepackage[bottom]{footmisc}
\usepackage{pifont}

%%% Local Variables: 
%%% mode: latex
%%% TeX-master: "../main.tex"
%%% End:

% 进行必要的设置
\input{setup/format.tex}

% 设置文档基本信息,\linebreak 前面不要有空格,否则在无需换行的场合,
% 中文之间的空格无法消除
% 另,在\nwafuset中不可以出现空行
\nwafuset{
  clscode = {TP391.9}, % 分类码,仅研究生需要
  udccode = {004.9},   % UDC码,仅研究生需要
  cfdlevel = {公开},   % 密级,仅研究生需要(只能取公开、限制、秘密、机密、绝密五个等级)
  unvcode = {10712},   % 学校代码,仅研究生需要,西北农林科技大学取10712
  studentid = {2013051289}, % 学号,本科/研究生需要
  gradyear = {2019}, % 毕业年,本科/研究生需要
  title = {\nwafuthesis{} 快速上手示例文档}, % 论文题目,本科/研究生需要
  professionaltype = {工程硕士},   % 类型,仅专硕需要(指金融硕士、社会工作硕士、工程硕士、农业推广硕士、兽医硕士、风景园林硕士、林业硕士、中药学硕士、工商管理硕士、公共管理硕士等10种类型。)
  professionalfield = {软件工程},  % 专硕,仅专硕需要
                                   % 农业推广硕士研究生填写:作物、园艺、农业资源利用、植物保护、养殖、草业、林业、
                                   % 渔业、农业机械化、农村与区域发展、农业科技组织与服务、农业信息化、食品加工与安全;
                                   % 
                                   % 工程硕士研究生填写:建筑与土木工程、水利工程、农业工程、环境工程、食品工程、生物工程等。
  majorsubject = {计算机应用技术},  % 学科专业,仅研究生需要
  researchfield = {智能媒体处理},   % 研究方向,仅研究生需要
  researchername = {\LaTeX{}er},   % 研究生论文作者姓名,仅研究生需要
  major = {计算机科学与技术},       % 专业,仅本科生需要
  advisers = {耿楠},               % 指导教师姓名,本科/研究生需要
  coadvisers = {Donald Knuth\quad 大师}, % 协助指导教师姓名,本科/研究生需要
  classid = {152},                 % 班级号,仅本科生需要(只填写数字,不要有其它内容)
  author = {\LaTeX{}er},           % 论文作者姓名,仅本科生需要
  college = {信息工程学院},         % 学院名称,仅本科生需要 
  applydate = {\today},            % 完成日期(默认为当前日期),本科/研究生需要
  defensedate = {\today},            % 答辩日期(默认为当前日期),研究生需要
  adviserteam = {耿楠,Knuth,Lamport}, % 指导小组,博士论文需要(不同人名用英文逗号分割)
  % 答辩委员会成员csv文件名称,包含相对路径,研究生论文需要
  cmteemembfile = {data/committeememb.csv},
  % 资助项目csv文件名称,包含相对路径,研究生论文需要
  ackdatafile = {data/ackdata.csv},
}

% 英文
\nwafusetEn{
  cfdlevel = {Open},
  title = {\nwafuthesis{} Quick Start and Document Snippets},% 论文英文题目,本科/研究生需要
  degreefull = {Philosophy},         % 学位类别,仅研究生需要
  majorsubject = {Computer Science}, % 学科专业,仅研究生需要
  researchfield = {Multimedia},      % 研究方向,仅研究生需要
  researchername = {\LaTeX{}er},     % 研究生论文作者姓名,仅研究生需要
  advisers = {Geng Nan},             % 指导教师姓名,本科/研究生需要
  coadvisers = {Donald Knuth, tex.se},  % 协助指导教师姓名,本科/研究生需要
  institute = {College of Information \& Engineering}
}

% biblatex宏包的参考文献数据源
\addbibresource[location=local]{bib/sample.bib}

\begin{document}

% 排版封面页
\makecover

% 排版资助声明页,研究生需要,无资助声明或本科论文请注释该行代码
\makeprojlist

% 排版独创性和版权声明页
\makedeclare

% 研究生需要将\frontmatter置于摘要之前,以实现页眉页脚处理
\frontmatter 
% 排版摘要
% 本文件是示例论文的一部分
% 论文的主文件位于上级目录的 `main.tex`

% 中文摘要
\begin{abstract}
在羽毛球比赛中,对羽毛球的运动轨迹进行分析对比赛结果的判定有着实际的意义。然而,问题是目前世界上并没有直接立体显示羽毛球运动轨迹的相机。如何提取和配准羽毛球的运动轨迹就是我的毕业设计要研究的内容。我在做毕业设计过程中,也会用到双目视觉技术。本文以羽毛球比赛中的羽毛球为研究对象,我需要对比赛场地进行分析和摄像机的标定。我要先识别运动的羽毛球,然后使用数字图像处理技术实现对运动的羽毛球的目标跟踪,以此来获取它的运动轨迹,再使用双目视觉技术对羽毛球运动轨迹进行配准,最后根据配准曲线来获得现实世界三维空间的实际轨迹。
%本文介绍如何使用\nwafuthesis{} 文档类撰写西北农林科技大学学位论文。

%首先介绍如何获取并编译本文档,然后展示论文部件的实例,最后列举部分常用宏包的使用方法。
\end{abstract}
% 中文关键词(用英文","分割)
\keywords{学位论文, 模板, \nwafuthesis}

% 英文摘要
\begin{abstractEn}
This document introduces \nwafuthesis, the \LaTeX{} document class for NWAFU Thesis.

First, we show how to get the source code and compile this document.
Then we provide snippets for figures, tables, equations, etc.
Finally we enforce some usage patterns.
\end{abstractEn}
% 英文文关键词(用英文","分割)
\keywordsEn{NWAFU thesis, document class, space is accepted here}

%%% Local Variables: 
%%% mode: latex
%%% TeX-master: "../main.tex"
%%% End:

\makeabstract

% 本科生需要将\frontmatter置于摘要之后,以实现空白页眉页脚处理
% \frontmatter
% 排版目录
% 如果需要调整目录层级数量的话,取消下一行注释,数字含义: 0=chapter, 1=section, 2=subsection
% \setcounter{tocdepth}{1}
\expandafter\nwafutableofcontents

%% 符号对照表, 本科生不需要
% \input{contents/denotation}

% 排版正文
\mainmatter
\include{contents/chap01}
\include{contents/chap02}
\include{contents/chap03}
\include{contents/chap04}
\include{contents/chap05}

% 排版参考文献表
\bibliomatter
\nocite{*}
%\nocite{广西壮族自治区林业厅1993--,r4,张若凌2004--,于潇2012-1518-1523,马克思2013-302-302,张田勤2000--,萧钰2001--,刘加林1993--,张志祥1998--,n42,n43}
\printbibliography[heading=bibintoc]%

% 排版附录(推荐用appendix环境排版)
\begin{appendix}
  \input{contents/chap06-app1}
  \input{contents/chap07-app2}
\end{appendix}

% 排版致谢
\backmatter
%% 致谢
% 如果使用声明扫描页,将可选参数指定为扫描后的 PDF 文件名,例如:
% \begin{acknowledgement}[scan-statement.pdf]
\begin{acknowledgement}

子在川上曰,逝者如斯夫,不舍昼夜。自吾去蜀入秦,凡五年矣。昔之来者,翩翩素衣,白马银鞍,谈笑无忌。今将去也,堪堪而立,褐面黄须,肱股生腴。不得少瑜之梦笔,唯学祖狄而闻鸡。心高气傲以格钛二铝铌之物,智短才疏稍致材料加工之知。为此浅陋之文,以资博士之谋,诚不胜惶恐也。

初入长安,即为恩师所知遇,幸何如之。恩师曾公,名讳上卫下东,少有才名。师夷西学,以涉重洋,修诸德国,而报故邦。求索未知,惟日孜孜,正襟治学,不尝稍忘。及至聘为教授,时年仅三十有四耳。潜心于经典,焚膏以继晷。学问博如四海,非唯囿于简牍。每亲临工厂,必鱼贯相请,凡所问者莫不相答。尝有经年不解之惑,观之如庖丁之牛,解之以经理,人皆称善,莫不拜服。吾师声名之隆者如此。自吾拜于门下,言传之,身教之,伏九不怠。及其斧正拙笔,字斟之,句酌之,晨昏弗懈。为学莫重于尊师,恩师循循以导,谆谆而教,恩德未可胜计,无论尽报。

予以二八之年求学于外,背井辗转已逾十年矣。进不得衣锦还乡,以光门庭,退未尝趋庭鲤对,而事双亲。其为子也,殊不孝也。人之行,莫大于孝。夫致孝者,怀橘卧冰,温衾恣蚊。无报严君之德,何如三迁之恩。吾素远游无方,岁末而归,十数日复去。独见故乡十年无夏,不察父母容颜渐改。父母年逾天命,两鬓霜凝,尤以垂垂之姿,而为版筑之作。每念及斯,愧也,疚也,恨无地也。吾弟求学于成都,学业既成,此诚不胜之喜也。幼时尾从终日,及长而别,少聚多离。愚兄痴长五岁,孝悌两违,贤弟勿见责也。

学贵得师,亦贵得友。朋曰共砚,友曰志同。承蒙见遇,铭诸五内。清风明月同唱苏子,高山流水共操五音。刀笔可录春秋,缣帛难表衷言。敬列诸君之名于文末,以表谢忱,倘有阙漏,唯乞见谅耳。

  感谢 \LaTeX 和 \nwafuthesis,帮我节省了不少时间。

西北工业大学博士\  郑友平

\newpage


东北大学信息科学与工程学院自动化专业2017届毕业生米威名花了三天的时间写了这篇致谢。致谢里,他感谢了母校和师长无微不至的关心与爱护以及母亲含辛茹苦的照顾。米威名现已保研清华大学自动化系。
先来欣赏一下理工科大神的文言文致谢吧!

致谢:

公元二千一七年,岁次丁酉,初夏之月,威名拙论乃告杀青。理微辞穷,未敢称凌云之作,镂心鸟迹,得不效相如之叹?于是凭窗啜饮,寄情遐思。

忆余初入东大,未及弱冠,书生意气,挥斥方遒,或废寝以搜读先哲,或忘食而亲验知行。浮云朝露,过隙白驹,距吾始书尔来已春秋有四,于今毕业,年齿已趋而立。户牅之外,万物滋荣,熙来攘往,景致阙如昨日;堂室之内,漫展书卷,激昂文字,然威名早已有苍颜白发矣。

文凭两纸霜鬓两行,黄粱一枕功名一场,此皆书生寻常,乏善可陈。然威名身蒙寸草春晖之恩情,春风化雨之陶冶,润物无声之教化,育诲之恩,重胜泰山,虽衔环结草不能报之万一。是以情造文,铭而致谢。

威名古襄平人氏,布衣世家,聿修祖德,孝悌累洽。襁褓之时,家徒四壁,父苦工在外,母荆钗持家,亏得亲邻接济方得度日,后父以技长,渐为小康。髫龀入蒙,受教庠序,趋庭鲤对,每日不辍。时吾腹诗三百,音字无差。本就天伦,然世无常,父猝而远去,唯留母子相濡。此近十载,吾母吐哺无稍息,咽苦不颦眉。蓼蓼者莪,匪莪伊蒿,欲报之德,昊天罔极!

及吾稍长,志求门楣光耀以报顾复,于是负笈求学,欢会长乖。闻道远行,慈母手线,怜儿夜寒。子在关山外,慈母念他乡。孔子曰,立身行道,以显父母;《诗经》云,夙兴夜寐,无忝尔所生。何有于威名哉!此威名胡跪而叩谢者一也。

吾校东大,国之成均。肇于九一八国难之将近,辗转十四载抗战之狼烟。溯源沈水,奄宅奉天。临清朝陪都宫殿之前庭,接民国张氏帅府之后坊。苍松掩路,翠柏当庭。宁图晨钟,央园月朗。俊彦迭代,济济一堂。自强不息以树帜,知行合一以闻章。

威名不才,三尺微命。薄德寡智,有辱斯文。母校慈垂,翼我缥囊。沐浴清化,问学课堂。克明畯德,知止后安。吾尝于宁恩承内,望书卷万轴,乃知科学之堂奥,人文之博深。吾尝于何世礼中,聆名家讲学,方觉大师之风范,匠心之精运。吾亦尝漫行于五五,听夜雨梧桐,泠泠作响,感四时寒暑之潜移,觉宇宙天地之苍凉,哀人生往来于须臾,叹砺志奋发以图强。母校恩养,没齿难忘。此威名胡跪而叩谢者二也。

余自入东大以来,累受师长教育之恩。恩师张先生云洲,温恭和蔼,德才兼具。于威名之所学,吾师循循善诱,发蒙启蔽,苦心孤诣,鱼渔双授;于威名之修身,吾师以身作则,行端表正,不言之教,桃下之蹊。吾辈性骄,常拒管教,师亦不弃嫌,呕心沥血,方有余今日之成。余心感念,早已视之如父。

而于本论文之撰写,自题目选定至文献查阅,自实验设计至机理探撷,自纲路结构至文段末节,皆得吾师贾子熙,导师张涛悉心指点,谢无尽焉。此间感科研之路漫漫,志当上下而求索。亦再恩导师张涛不厌吾愚,允余北面承贽,以沐清华之泽,承先辈弦歌,勉夙愿之怀,此桃李之恩,片纸难详。《诗》曰:赫赫师尹,民具尔瞻。歌曰:云山苍苍,江水泱泱。先生之风,山高水长。艟艨巨舰,非桨舵导引之助不能乘风破浪;北溟鲲鹏,非长风托举之力不能垂翼九天。此威名胡跪而叩谢者三也。

诚惶诚恐,飏拜稽首。




\end{acknowledgement}


%% 个人简历
\input{contents/resume}

\end{document}

%%% Local Variables:
%%% mode: latex
%%% TeX-master: t
%%% End:
